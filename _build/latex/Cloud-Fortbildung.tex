% Generated by Sphinx.
\def\sphinxdocclass{report}
\documentclass[letterpaper,10pt,ngerman]{sphinxmanual}
\usepackage[utf8]{inputenc}
\DeclareUnicodeCharacter{00A0}{\nobreakspace}
\usepackage{cmap}
\usepackage[T1]{fontenc}

\usepackage{babel}
\usepackage{times}
\usepackage[Sonny]{fncychap}
\usepackage{longtable}
\usepackage{sphinx}
\usepackage{multirow}
\usepackage{eqparbox}


\addto\captionsngerman{\renewcommand{\figurename}{Abb. }}
\addto\captionsngerman{\renewcommand{\tablename}{Tab. }}
\SetupFloatingEnvironment{literal-block}{name=Quellcode }



\title{Cloud-Fortbildung Documentation}
\date{27.11.2017}
\release{1}
\author{Christian Huber}
\newcommand{\sphinxlogo}{}
\renewcommand{\releasename}{Release}
\setcounter{tocdepth}{1}
\makeindex

\makeatletter
\def\PYG@reset{\let\PYG@it=\relax \let\PYG@bf=\relax%
    \let\PYG@ul=\relax \let\PYG@tc=\relax%
    \let\PYG@bc=\relax \let\PYG@ff=\relax}
\def\PYG@tok#1{\csname PYG@tok@#1\endcsname}
\def\PYG@toks#1+{\ifx\relax#1\empty\else%
    \PYG@tok{#1}\expandafter\PYG@toks\fi}
\def\PYG@do#1{\PYG@bc{\PYG@tc{\PYG@ul{%
    \PYG@it{\PYG@bf{\PYG@ff{#1}}}}}}}
\def\PYG#1#2{\PYG@reset\PYG@toks#1+\relax+\PYG@do{#2}}

\expandafter\def\csname PYG@tok@gd\endcsname{\def\PYG@tc##1{\textcolor[rgb]{0.63,0.00,0.00}{##1}}}
\expandafter\def\csname PYG@tok@gu\endcsname{\let\PYG@bf=\textbf\def\PYG@tc##1{\textcolor[rgb]{0.50,0.00,0.50}{##1}}}
\expandafter\def\csname PYG@tok@gt\endcsname{\def\PYG@tc##1{\textcolor[rgb]{0.00,0.27,0.87}{##1}}}
\expandafter\def\csname PYG@tok@gs\endcsname{\let\PYG@bf=\textbf}
\expandafter\def\csname PYG@tok@gr\endcsname{\def\PYG@tc##1{\textcolor[rgb]{1.00,0.00,0.00}{##1}}}
\expandafter\def\csname PYG@tok@cm\endcsname{\let\PYG@it=\textit\def\PYG@tc##1{\textcolor[rgb]{0.25,0.50,0.56}{##1}}}
\expandafter\def\csname PYG@tok@vg\endcsname{\def\PYG@tc##1{\textcolor[rgb]{0.73,0.38,0.84}{##1}}}
\expandafter\def\csname PYG@tok@vi\endcsname{\def\PYG@tc##1{\textcolor[rgb]{0.73,0.38,0.84}{##1}}}
\expandafter\def\csname PYG@tok@mh\endcsname{\def\PYG@tc##1{\textcolor[rgb]{0.13,0.50,0.31}{##1}}}
\expandafter\def\csname PYG@tok@cs\endcsname{\def\PYG@tc##1{\textcolor[rgb]{0.25,0.50,0.56}{##1}}\def\PYG@bc##1{\setlength{\fboxsep}{0pt}\colorbox[rgb]{1.00,0.94,0.94}{\strut ##1}}}
\expandafter\def\csname PYG@tok@ge\endcsname{\let\PYG@it=\textit}
\expandafter\def\csname PYG@tok@vc\endcsname{\def\PYG@tc##1{\textcolor[rgb]{0.73,0.38,0.84}{##1}}}
\expandafter\def\csname PYG@tok@il\endcsname{\def\PYG@tc##1{\textcolor[rgb]{0.13,0.50,0.31}{##1}}}
\expandafter\def\csname PYG@tok@go\endcsname{\def\PYG@tc##1{\textcolor[rgb]{0.20,0.20,0.20}{##1}}}
\expandafter\def\csname PYG@tok@cp\endcsname{\def\PYG@tc##1{\textcolor[rgb]{0.00,0.44,0.13}{##1}}}
\expandafter\def\csname PYG@tok@gi\endcsname{\def\PYG@tc##1{\textcolor[rgb]{0.00,0.63,0.00}{##1}}}
\expandafter\def\csname PYG@tok@gh\endcsname{\let\PYG@bf=\textbf\def\PYG@tc##1{\textcolor[rgb]{0.00,0.00,0.50}{##1}}}
\expandafter\def\csname PYG@tok@ni\endcsname{\let\PYG@bf=\textbf\def\PYG@tc##1{\textcolor[rgb]{0.84,0.33,0.22}{##1}}}
\expandafter\def\csname PYG@tok@nl\endcsname{\let\PYG@bf=\textbf\def\PYG@tc##1{\textcolor[rgb]{0.00,0.13,0.44}{##1}}}
\expandafter\def\csname PYG@tok@nn\endcsname{\let\PYG@bf=\textbf\def\PYG@tc##1{\textcolor[rgb]{0.05,0.52,0.71}{##1}}}
\expandafter\def\csname PYG@tok@no\endcsname{\def\PYG@tc##1{\textcolor[rgb]{0.38,0.68,0.84}{##1}}}
\expandafter\def\csname PYG@tok@na\endcsname{\def\PYG@tc##1{\textcolor[rgb]{0.25,0.44,0.63}{##1}}}
\expandafter\def\csname PYG@tok@nb\endcsname{\def\PYG@tc##1{\textcolor[rgb]{0.00,0.44,0.13}{##1}}}
\expandafter\def\csname PYG@tok@nc\endcsname{\let\PYG@bf=\textbf\def\PYG@tc##1{\textcolor[rgb]{0.05,0.52,0.71}{##1}}}
\expandafter\def\csname PYG@tok@nd\endcsname{\let\PYG@bf=\textbf\def\PYG@tc##1{\textcolor[rgb]{0.33,0.33,0.33}{##1}}}
\expandafter\def\csname PYG@tok@ne\endcsname{\def\PYG@tc##1{\textcolor[rgb]{0.00,0.44,0.13}{##1}}}
\expandafter\def\csname PYG@tok@nf\endcsname{\def\PYG@tc##1{\textcolor[rgb]{0.02,0.16,0.49}{##1}}}
\expandafter\def\csname PYG@tok@si\endcsname{\let\PYG@it=\textit\def\PYG@tc##1{\textcolor[rgb]{0.44,0.63,0.82}{##1}}}
\expandafter\def\csname PYG@tok@s2\endcsname{\def\PYG@tc##1{\textcolor[rgb]{0.25,0.44,0.63}{##1}}}
\expandafter\def\csname PYG@tok@nt\endcsname{\let\PYG@bf=\textbf\def\PYG@tc##1{\textcolor[rgb]{0.02,0.16,0.45}{##1}}}
\expandafter\def\csname PYG@tok@nv\endcsname{\def\PYG@tc##1{\textcolor[rgb]{0.73,0.38,0.84}{##1}}}
\expandafter\def\csname PYG@tok@s1\endcsname{\def\PYG@tc##1{\textcolor[rgb]{0.25,0.44,0.63}{##1}}}
\expandafter\def\csname PYG@tok@ch\endcsname{\let\PYG@it=\textit\def\PYG@tc##1{\textcolor[rgb]{0.25,0.50,0.56}{##1}}}
\expandafter\def\csname PYG@tok@m\endcsname{\def\PYG@tc##1{\textcolor[rgb]{0.13,0.50,0.31}{##1}}}
\expandafter\def\csname PYG@tok@gp\endcsname{\let\PYG@bf=\textbf\def\PYG@tc##1{\textcolor[rgb]{0.78,0.36,0.04}{##1}}}
\expandafter\def\csname PYG@tok@sh\endcsname{\def\PYG@tc##1{\textcolor[rgb]{0.25,0.44,0.63}{##1}}}
\expandafter\def\csname PYG@tok@ow\endcsname{\let\PYG@bf=\textbf\def\PYG@tc##1{\textcolor[rgb]{0.00,0.44,0.13}{##1}}}
\expandafter\def\csname PYG@tok@sx\endcsname{\def\PYG@tc##1{\textcolor[rgb]{0.78,0.36,0.04}{##1}}}
\expandafter\def\csname PYG@tok@bp\endcsname{\def\PYG@tc##1{\textcolor[rgb]{0.00,0.44,0.13}{##1}}}
\expandafter\def\csname PYG@tok@c1\endcsname{\let\PYG@it=\textit\def\PYG@tc##1{\textcolor[rgb]{0.25,0.50,0.56}{##1}}}
\expandafter\def\csname PYG@tok@o\endcsname{\def\PYG@tc##1{\textcolor[rgb]{0.40,0.40,0.40}{##1}}}
\expandafter\def\csname PYG@tok@kc\endcsname{\let\PYG@bf=\textbf\def\PYG@tc##1{\textcolor[rgb]{0.00,0.44,0.13}{##1}}}
\expandafter\def\csname PYG@tok@c\endcsname{\let\PYG@it=\textit\def\PYG@tc##1{\textcolor[rgb]{0.25,0.50,0.56}{##1}}}
\expandafter\def\csname PYG@tok@mf\endcsname{\def\PYG@tc##1{\textcolor[rgb]{0.13,0.50,0.31}{##1}}}
\expandafter\def\csname PYG@tok@err\endcsname{\def\PYG@bc##1{\setlength{\fboxsep}{0pt}\fcolorbox[rgb]{1.00,0.00,0.00}{1,1,1}{\strut ##1}}}
\expandafter\def\csname PYG@tok@mb\endcsname{\def\PYG@tc##1{\textcolor[rgb]{0.13,0.50,0.31}{##1}}}
\expandafter\def\csname PYG@tok@ss\endcsname{\def\PYG@tc##1{\textcolor[rgb]{0.32,0.47,0.09}{##1}}}
\expandafter\def\csname PYG@tok@sr\endcsname{\def\PYG@tc##1{\textcolor[rgb]{0.14,0.33,0.53}{##1}}}
\expandafter\def\csname PYG@tok@mo\endcsname{\def\PYG@tc##1{\textcolor[rgb]{0.13,0.50,0.31}{##1}}}
\expandafter\def\csname PYG@tok@kd\endcsname{\let\PYG@bf=\textbf\def\PYG@tc##1{\textcolor[rgb]{0.00,0.44,0.13}{##1}}}
\expandafter\def\csname PYG@tok@mi\endcsname{\def\PYG@tc##1{\textcolor[rgb]{0.13,0.50,0.31}{##1}}}
\expandafter\def\csname PYG@tok@kn\endcsname{\let\PYG@bf=\textbf\def\PYG@tc##1{\textcolor[rgb]{0.00,0.44,0.13}{##1}}}
\expandafter\def\csname PYG@tok@cpf\endcsname{\let\PYG@it=\textit\def\PYG@tc##1{\textcolor[rgb]{0.25,0.50,0.56}{##1}}}
\expandafter\def\csname PYG@tok@kr\endcsname{\let\PYG@bf=\textbf\def\PYG@tc##1{\textcolor[rgb]{0.00,0.44,0.13}{##1}}}
\expandafter\def\csname PYG@tok@s\endcsname{\def\PYG@tc##1{\textcolor[rgb]{0.25,0.44,0.63}{##1}}}
\expandafter\def\csname PYG@tok@kp\endcsname{\def\PYG@tc##1{\textcolor[rgb]{0.00,0.44,0.13}{##1}}}
\expandafter\def\csname PYG@tok@w\endcsname{\def\PYG@tc##1{\textcolor[rgb]{0.73,0.73,0.73}{##1}}}
\expandafter\def\csname PYG@tok@kt\endcsname{\def\PYG@tc##1{\textcolor[rgb]{0.56,0.13,0.00}{##1}}}
\expandafter\def\csname PYG@tok@sc\endcsname{\def\PYG@tc##1{\textcolor[rgb]{0.25,0.44,0.63}{##1}}}
\expandafter\def\csname PYG@tok@sb\endcsname{\def\PYG@tc##1{\textcolor[rgb]{0.25,0.44,0.63}{##1}}}
\expandafter\def\csname PYG@tok@k\endcsname{\let\PYG@bf=\textbf\def\PYG@tc##1{\textcolor[rgb]{0.00,0.44,0.13}{##1}}}
\expandafter\def\csname PYG@tok@se\endcsname{\let\PYG@bf=\textbf\def\PYG@tc##1{\textcolor[rgb]{0.25,0.44,0.63}{##1}}}
\expandafter\def\csname PYG@tok@sd\endcsname{\let\PYG@it=\textit\def\PYG@tc##1{\textcolor[rgb]{0.25,0.44,0.63}{##1}}}

\def\PYGZbs{\char`\\}
\def\PYGZus{\char`\_}
\def\PYGZob{\char`\{}
\def\PYGZcb{\char`\}}
\def\PYGZca{\char`\^}
\def\PYGZam{\char`\&}
\def\PYGZlt{\char`\<}
\def\PYGZgt{\char`\>}
\def\PYGZsh{\char`\#}
\def\PYGZpc{\char`\%}
\def\PYGZdl{\char`\$}
\def\PYGZhy{\char`\-}
\def\PYGZsq{\char`\'}
\def\PYGZdq{\char`\"}
\def\PYGZti{\char`\~}
% for compatibility with earlier versions
\def\PYGZat{@}
\def\PYGZlb{[}
\def\PYGZrb{]}
\makeatother

\renewcommand\PYGZsq{\textquotesingle}

\begin{document}
\shorthandoff{"}
\maketitle
\tableofcontents
\phantomsection\label{index::doc}


Contents:


\chapter{Zugriff auf die Cloud}
\label{Zugriff auf die Cloud:welcome-to-cloud-fobi-s-documentation}\label{Zugriff auf die Cloud::doc}\label{Zugriff auf die Cloud:zugriff-auf-die-cloud}\phantomsection\label{Zugriff auf die Cloud:zugriff-sk}
Zugriff auf die Cloud im Browser erhalten Sie, indem sie auf der Startseite der schuleigenen Homepage auf den Link „copbox“ klicken oder im Browser direkt folgende URL eingeben:

\href{https://owncloud.copernicus-gymnasium.de/owncloud}{https://owncloud.copernicus-gymnasium.de/owncloud}

Dort können Sie sich mit ihren Accountdaten anmelden.


\section{Die Startseite}
\label{Zugriff auf die Cloud:startseite-cloud-sk}\label{Zugriff auf die Cloud:die-startseite}
Nach der erfolgreichen Anmeldung im Browser gelangen sie zur Startseite der Cloud. Hier sehen sie ihr Cloudverzeichnis und die darin enthaltenen Dateien und Ordner.

\includegraphics{{41}.png}

Den meisten Platz nimmt die Übersicht über die Ordner und Dateien, die sich in ihrem Cloudverzeichnis befinden (1.).
Weiterhin sehen sie am linken Rand einen Bereich, der Ihnen bei Bedarf genauere Informationen über die von oder mit Ihnen geteilten Dateien bietet (2.).
Einen schnellen Überblick, wer Ihnen eine Datei oder einen Ordner geteilt hat, finden Sie rechts neben den Dateien/Ordnern. Steht da lediglich ``Geteilt'', klicken
Sie auf dieses Wort, um genauere Informationen zu erhalten.
Links oben neben dem Cloud-Symbol befinden sich die verschiedenen Anwendungen (``Apps''), die Ihnen die Cloud bietet(3.).

Sie sehen folgende Bereiche (von links nach rechts):
\begin{itemize}
\item {} 
``Dateien'': bringt Sie zurück zur Übersicht ihrer Dateien und Ordner

\item {} 
``Aktivität'': hier finden sie Informationen über die mit oder von Ihnen geteilten Dateien und wer, wann, was damit gemacht hat.

\item {} 
``Galerie'': eine Sammlung aller Bilder, die sie auf der Cloud gespeichert haben.

\item {} 
``Kreise'': bietet die Möglichkeit, Gruppen zu erstellen, um Dateien einfach mit mehreren Benutzern zu teilen.

\item {} 
``Kalender'': wie der Name schon sagt, können Sie hier einen Kalender führen, andere Kalender einbinden, diese Kalender mit anderen teilen oder mit den Geräte zu Hause synchronisieren.

\item {} 
``Audio-Player'': bietet Ihnen die Möglichkeit, Audio-Dateien direkt aus der Cloud heraus abzuspielen.

\item {} 
``Notizen'': Raum für Notizen, Gedanken, Einfälle

\item {} 
``Ankündigungen'': Hier finden Sie Mitteilungen des Admins, die sicherlich immer einen Blick wert sind.

\end{itemize}

Oben rechts auf der Startseite können Sie nach einem Klick auf die Lupe nach Dateien suchen.
Rechts daneben sehen Sie eine Schaltfläche mit einer Glocke. Hier verbergen sich Ankündigungen des Administrators oder sonstige Benachrichtigungen.
Ein Klick auf das Icon mit den beiden angedeuteten Personen zeigt Ihnen alle Benutzer der Cloud an, mit denen Sie Dateien geteilt haben.
Ganz rechts bringt Sie ein Klick auf das Zahnrad zu den Einstellungen ihres Nutzerprofils (4.)

Unten links sehen Sie die Schaltfläche ``Gelöschte Dateien''. Hier verbergen Sie von Ihnen gelöschte Dateien, die sie bei Bedarf wiederherstellen können.
Darunter wird Ihnen die aktuelle Belegung Ihres Cloudspeichers angezeigt.
Ganz unten finden Sie den Punkt ``Einstellungen''. Ein Klick darauf gibt einerseits die Möglichkeit versteckte Dateien in Ihrer Cloud anzuzeigen, andereseits wird hier auch ein
WebDav-Link zu ihrem Cloudspeicher erzeugt, mit dessen Hilfe es möglich ist, Ihr Cloudverzeichnis in andere Geräte einzubinden (5.).


\chapter{Kalender}
\label{Kalender::doc}\label{Kalender:kalender}

\section{Eine Klassenarbeit oder eine Exkursion eintragen}
\label{Kalender:kalender-sk}\label{Kalender:eine-klassenarbeit-oder-eine-exkursion-eintragen}
Seit Schuljahresbeginn (2017/18) existieren für die Lehrer mehrere Kalender in der Cloud, in die jeder seine Klassenarbeiten und Exkursionen
eintragen kann.

Für jede Klasse bzw. Klassenstufe wurde ein eigener Kalender erstellt! Damit die Übersichtlichkeit gewahrt bleibt, kann man alle Kalender, die man nicht zu sehen wünscht, mit einem Klick
auf den farbigen Punkt links neben dem Kalendernamen, ausblenden. Übrig bleiben in der Terminübersicht rechts dann nur die farbig markierten Kalender.

Der Vorteil liegt auf der Hand: Man kann in Ruhe zu Hause seine Klassenarbeit planen und eintragen, da die Kalender rund um die Uhr auch von
zu Hause (oder von anderswo) aus erreichbar sind. Man kann diese Kalender bequem in ein Email-Programm mit Kalenderfunktion einbinden (Thunderbird, Outlook, Apple-Variante)
oder auch in eine entsprechende App einbinden, um die Kalender auch auf mobilen Geräten verfügbar zu machen.

\begin{notice}{important}{Wichtig:}
Allerdings sollten Sie folgende Punkte bedenken:  Der Netzwerkberater stellt die Funktionalität bereit und ist nicht für fehlerhafte oder nicht korrekt angelegte Einträge verantwortlich. Vergewissern Sie sich daher bitte immer,a) ob der Inhalt des Eintrags korrekt ist, b) der Eintrag in dem richtigen Kalender (``Arbeiten u. Exkursionen'') angelegt wurde und c), dass der Eintrag gespeichtert wurde. Bedenken Sie weiterhin, dass Sie die Einträge der Kollegen verändern und löschen können. Bitte vermeiden Sie dies unbedingt.
\end{notice}

Vorgehen:
\begin{enumerate}
\item {} 
Klicken Sie Hauptfenster in der Leiste der Apps auf das Kalendersymbol

\end{enumerate}

\includegraphics{{11}.png}

2. Sie werden zur Kalender-App weitergeleitet. Die meisten Platz auf dem Bildschirm nimmt die Kalenderübersicht ein. Ein Kästchen pro Tag.
Links oben sehen sie den angezeigten Monat und das Jahr. Klicken Sie auf die Pfeile links oder rechts neben Anzeige von Monat und Jahr, um zum gewünschten Termin zu gelangen.
Klicken Sie auf die Schaltflächen ``Tag'', ``Woche'', ``Monat'' oder ``Heute'', um die Ansicht zu ändern. Die Übersicht rechts verändert sich entsprechend.
Weiterhin sehen Sie links alle Kalender die sie angelegt haben, oder die mit Ihnen geteilt wurden. Das sind wahrscheinlich nur zwei. Ein blau markierter mit dem Namen ``Persönlich'' für ihre eigenen Termine und einen rot markierten darunter mit dem Namen ``Arbeiten und Exkursionen''.

\includegraphics{{12}.png}

3. Um eine Klassenarbeit oder eine Exkursion in diesem Kalender einzutragen, doppelklicken Sie auf das gewünschte Datum in der Übersicht.
Es öffnet sich ein Fenster, in das sie bitte den Namen des Eintrags eintragen (``Titel der Veranstaltung'') sowie die Dauer (die Uhrzeit).

\begin{notice}{important}{Wichtig:}
Um die Übersichtlichkeit zu erhöhen und um für ein einheitliches Erscheinungsbild zu sorgen halten Sie sich bitte an folgende Schreibweise: Klasse,Fach,Lehrerkürzel. Zum Beispiel: 9d,D,Hu
\end{notice}

\includegraphics{{13}.png}

\begin{notice}{important}{Wichtig:}
Achten Sie darauf, dass Sie den richtigen Kalender ausgewählt haben. Klicken Sie auf das Dropdown-Menu und wählen Sie den Kalender derjenigen Klasse, in den Sie eine Klassenarbeit oder eine Exkursion eintragen möchten, z. B. ``6a''.
\end{notice}

Sind alle Angaben korrekt und der richtige Kalender ausgewählt, klicken Sie auf die Schaltfläche ``Erstellen'' unten rechts.

\includegraphics{{14}.png}

Wenn alles geklappt hat, erscheint der Termin in der Kalenderübersicht in roter Farbe! Der Termin ist nun für alle anderen Kollegen sichtbar.

\includegraphics{{15}.png}


\section{Aufgaben}
\label{Kalender:aufgaben}\begin{enumerate}
\item {} 
Gehen Sie zur Kalender-App!

\item {} 
Lassen Sie sich nur die Kalender anzeigen, in deren Klassen Sie unterrichten, sowie den Kalender ``Fortbildung''.

\item {} 
Erstellen Sie einen Termin im Kalender ``Fortbildung'' in der Woche vom 4.12. bis 10.12., der 45 Minuten dauert und folgendem Schema folgt: Klasse,Fach,Kürzel (welche Klasse Sie wählen, ist egal)

\item {} 
Überprüfen Sie, ob die anderen Teilnehmer Ihren Termin sehen können und ob diese Ihren Termin angezeigt bekommen.

\end{enumerate}


\chapter{Dateien}
\label{Dateien hochladen::doc}\label{Dateien hochladen:dateien}

\section{Dateien hoch- und herunterladen.}
\label{Dateien hochladen:dateien-hochladen-sk}\label{Dateien hochladen:dateien-hoch-und-herunterladen}\begin{enumerate}
\item {} 
Klicken sie einmal auf das Plus unterhalb des blauen Balkens

\item {} 
Klicken Sie einmal auf ``Hochladen''

\end{enumerate}

\includegraphics{{5}.png}
\begin{enumerate}
\setcounter{enumi}{2}
\item {} 
Wählen Sie in dem sich öffnenden Fenster die Datei(en) aus, die sie hochladen möchten (Sie können auch mehrere Dateien im selben Ordner auf einmal auswählen, indem Sie die ``Shift-Taste'' gedrückt halten und mit den Pfeiltasten die entsprechenden Dateien auswählen).

\end{enumerate}

\includegraphics{{upload2}.png}
\begin{enumerate}
\setcounter{enumi}{3}
\item {} 
Klicken Sie anschließend auf ``öffnen''.

\item {} 
Sie sehen nun einen blauen Balken, der den Fortschritt des Uploads anzeigt. Je nach Größe der Datei(en) kann der Upload eine Weile dauern. Haben Sie Geduld.

\end{enumerate}

\includegraphics{{upload3}.png}
\begin{enumerate}
\setcounter{enumi}{4}
\item {} 
Sie können auch einfach Dateien per ``Drag\&Drop'' vom PC in das Browserfenster ziehen, um Dateien in die Cloud hochzuladen.

\end{enumerate}

\includegraphics{{upload4}.png}


\section{Mehrere Dateien und Ordner auswählen}
\label{Dateien hochladen:mehrere-dateien-und-ordner-auswahlen}
Um den workflow zu verbessern können, Sie auch mit mehreren Dateien gleichzeitig arbeiten.
\begin{enumerate}
\item {} 
Bewegen Sie die Maus über die Dateien. Die ganze Zeile in der Übersicht erscheint nur leicht eingegraut.

\end{enumerate}

\includegraphics{{mehrere-dateien1}.png}
\begin{enumerate}
\setcounter{enumi}{1}
\item {} 
Sie werden feststellen, dass sich die Anzeige des Icons verändert hat und nun am rechten Rand der Datei oder des Ordners ein kleines leeres Kästchen zu sehen ist.

\item {} 
Klicken Sie in dieses Kästchen und es wird mit einem Haken markiert.

\item {} 
Über der Liste mit Dateien wird nun angezeigt, was Sie alles markiert haben. Dort ist auch ein weiteres leeres Kästchen zu sehen. Wenn Sie dieses anklicken, werden automatisch alle Dateien und Ordner in der Übersicht markiert. So lassen sich viele Dateien auf einmal herunterladen oder teilen.

\end{enumerate}


\section{Aufgaben:}
\label{Dateien hochladen:aufgaben}\begin{enumerate}
\item {} 
Erstellen Sie auf ihrem Computer eine Datei mit dem Namen ``test''.

\item {} 
Laden Sie diese Datei in die Cloud. Probieren Sie beide Varianten aus.

\end{enumerate}


\chapter{Dateien erstellen}
\label{Dateien erstellen:dateien-erstellen}\label{Dateien erstellen::doc}

\section{Erstellen von Dateien und Ordnern in der Cloud}
\label{Dateien erstellen:erstellen-von-dateien-und-ordnern-in-der-cloud}\label{Dateien erstellen:erstellen-sk}
Sie können Ordner und Dateien (Textdateien) auch direkt in der Cloud erstellen und bearbeiten.
\begin{enumerate}
\item {} 
Klicken Sie auf das + Symbol.

\item {} 
Klicken Sie auf ``Ordner'' oder ``Textdatei''.

\item {} 
Geben Sie den Namen des Ordners oder der Textdatei an.

\end{enumerate}

Der neu erstellte Ordner oder die angelegte Datei erscheint kurz darauf in der Übersicht ihrer Dateien.
Möchten Sie eine Datei in einem bestimmten Ordner verschieben, können Sie das per Drag\&Drop erledigen.


\section{Aufgaben:}
\label{Dateien erstellen:aufgaben}\begin{enumerate}
\item {} 
Erstellen Sie in der Cloud eine Datei mit einem beliebigen Namen!

\item {} 
Versuchen Sie, diese Datei umzubenennen.

\item {} 
Bitte diese Datei nicht löschen. Wir brauchen sie noch.

\end{enumerate}


\chapter{Dateien teilen}
\label{Dateien teilen:dateien-teilen}\label{Dateien teilen::doc}

\section{Dateien mit anderen teilen}
\label{Dateien teilen:dateien-mit-anderen-teilen}\label{Dateien teilen:id1}
Auf der Startseite sehen sie alle Dateien und Ordner, die sich in ihrem Cloud-Verzeichnis befinden. Sie sehen auch,
wer Dateien mit Ihnen geteilt hat. (1.)
Um Dateien und Ordner mit anderen Personen oder Gruppen zu teilen, klicken auf das Teilen-Symbol (2.).

\includegraphics{{1}.png}

Es öffnet sich rechts ein Bereich, der Detail des zu teilenden Objekts enthält.

\includegraphics{{2}.png}

Um eine Datei oder einen Ordner zu teilen, stellen Sie sicher, dass das Wort ``Teilen'' markiert (fett und unterstrichen) ist (1.).
Falls Sie eine Bemerkung zu der ausgewählten Datei abgeben möchten, klicken Sie auf ``Kommentare''. Personen, mit denen Sie
diese Datei teilen, können diesen Kommentar sehen.
Haben Sie auf ``Teilen'' geklickt, sehen Sie ein Eingabefeld, in das Sie den Namen der Person oder Gruppe eintragen, mit dem Sie die
Datei(en) oder Ordner teilen möchten (2.).
Erscheint der richtige Name, klicken Sie auf ihn und die Datei oder der Ordner wird mit der entsprechenden Person oder Gruppe geteilt.
Setzen Sie den Haken in der Box ``Link teilen'', wenn Sie nur den Link zu einem bestimmten Objekt teilen wollen (3.). Den auftauchenden Link können Sie z. B.
per Email verschicken. Der Empfänger kann dann auf das verlinkte Objekt zugreifen. Praktisch, wenn Sie eine Datei mit jemandem teilen möchten, der sonst keinen Zugang
zu dieser Cloud hat.

\includegraphics{{3}.png}

\begin{notice}{important}{Wichtig:}
Wenn Sie NICHT möchten, dass die Person, mit denen Sie ein Objekt geteilt haben, dieses bearbeiten kann, dann entfernen Sie den Haken neben ``kann bearbeiten''.
\end{notice}

Neben dem Namen finden Sie weitere Optionen. Setzen Sie den Haken, wenn Sie ein Bearbeiten der Datei etc. erlauben möchten.
Wenn Sie dann auf die drei Punkte neben Namen der Person, die etwas geteilt bekommt, klicken, öffnet sich ein Fenster mit verschiedenen Optionen,
die es Ihnen erlauben zu bestimmen, was die andere Person mit dem Objekt anstellen darf. Setzen Sie die Haken, wie es Ihnen am besten erscheint.

\includegraphics{{4}.png}


\section{Die Teilung wieder aufheben}
\label{Dateien teilen:die-teilung-wieder-aufheben}
Haben Sie versehentlich die falsche Datei geteilt oder etwas mit der falschen Person oder Gruppe geteilt, haben Sie in den Optionen, die sich hinter den drei Punkten
verbergen, die Möglichkeit, die Teilung wieder rückkgängig zu machen. Klicken Sie einfach auf den letzten Eintrag in der Liste der Optionen ``Freigabe aufheben''.

\includegraphics{{51}.png}

Das geteilte Objekt verschwindet daraufhin aus dem Cloudverzeichnis der Person(en), mit der Sie es geteilt haben.


\section{Aufgaben}
\label{Dateien teilen:aufgaben}
1. Teilen Sie die Datei, die Sie vorhin angelegt haben, mit einem Teilnehmer der Fortbilung und mit Huber.
Achten Sie darauf, dass diejengen, mit denen Sie diese Datei teilen, diese NICHT bearbeiten können.
\begin{enumerate}
\setcounter{enumi}{1}
\item {} 
Machen Sie die Freigabe für Huber wieder rückgängig!

\end{enumerate}


\chapter{Indices and tables}
\label{index:indices-and-tables}\begin{itemize}
\item {} 
\DUspan{xref,std,std-ref}{genindex}

\item {} 
\DUspan{xref,std,std-ref}{modindex}

\item {} 
\DUspan{xref,std,std-ref}{search}

\end{itemize}



\renewcommand{\indexname}{Stichwortverzeichnis}
\printindex
\end{document}
